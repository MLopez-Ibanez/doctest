\nonstopmode{}
\documentclass[letterpaper]{book}
\usepackage[times,inconsolata,hyper]{Rd}
\usepackage{makeidx}
\usepackage[utf8]{inputenc} % @SET ENCODING@
% \usepackage{graphicx} % @USE GRAPHICX@
\makeindex{}
\begin{document}
\chapter*{}
\begin{center}
{\textbf{\huge Package `doctest'}}
\par\bigskip{\large \today}
\end{center}
\inputencoding{utf8}
\ifthenelse{\boolean{Rd@use@hyper}}{\hypersetup{pdftitle = {doctest: Generate Tests from Examples Using 'roxygen' and 'testthat'}}}{}
\ifthenelse{\boolean{Rd@use@hyper}}{\hypersetup{pdfauthor = {David Hugh-Jones}}}{}
\begin{description}
\raggedright{}
\item[Type]\AsIs{Package}
\item[Title]\AsIs{Generate Tests from Examples Using 'roxygen' and 'testthat'}
\item[Version]\AsIs{0.3.0}
\item[Maintainer]\AsIs{David Hugh-Jones }\email{davidhughjones@gmail.com}\AsIs{}
\item[Description]\AsIs{Creates 'testthat' tests from 'roxygen' examples using simple tags.}
\item[License]\AsIs{MIT + file LICENSE}
\item[Encoding]\AsIs{UTF-8}
\item[RoxygenNote]\AsIs{7.2.3}
\item[Imports]\AsIs{cli, fs, pkgload, purrr, rlang, roxygen2, testthat}
\item[Suggests]\AsIs{covr, knitr, lifecycle, rmarkdown}
\item[Config/testthat/edition]\AsIs{3}
\item[Roxygen]\AsIs{list(markdown = TRUE)}
\item[URL]\AsIs{}\url{https://hughjonesd.github.io/doctest/}\AsIs{}
\item[BugReports]\AsIs{}\url{https://github.com/hughjonesd/doctest/issues}\AsIs{}
\item[VignetteBuilder]\AsIs{knitr}
\item[NeedsCompilation]\AsIs{no}
\item[Author]\AsIs{David Hugh-Jones [aut, cre]}
\end{description}
\Rdcontents{\R{} topics documented:}
\inputencoding{utf8}
\HeaderA{doctest-package}{Write testthat tests for your examples, using roxygen tags}{doctest.Rdash.package}
\aliasA{doctest}{doctest-package}{doctest}
%
\begin{Description}
The doctest package lets you test the code in your "Examples"
section in .Rd files. It uses the roxygen2 and testthat packages.
For more information, see \LinkA{@doctest}{@doctest} and \LinkA{@expect}{@expect}.
\end{Description}
%
\begin{Details}
%
\begin{SubSection}{Example}

Here's some \Rhref{https://roxygen2.r-lib.org}{roxygen} documentation for a function:

\begin{alltt}
#' Fibonacci function 
#' 
#' @param n Integer
#' @return The nth Fibonacci number
#' 
#' @doctest
#'
#' @expect type("integer")
#' fib(2)
#'
#' n <- 6 
#' @expect equal(8)
#' fib(n)
#' 
#' @expect warning("not numeric")
#' fib("a")
#'
#' @expect warning("NA")
#' fib(NA)
fib <- function (n) \{
  if (! is.numeric(n)) warning("n is not numeric")
  ...
\}
\end{alltt}


Instead of an \AsIs{\texttt{@examples}} section, we have a \AsIs{\texttt{@doctest}} section.

This will create tests like:

\begin{alltt}# Generated by doctest: do not edit by hand
# Please edit file in R/<text>

test_that("Doctest: fib", \{
  # Created from @doctest for `fib`
  # Source file: <text>
  # Source line: 7
  expect_type(fib(2), "integer")
  n <- 6
  expect_equal(fib(n), 8)
  expect_warning(fib("a"), "not numeric")
  expect_warning(fib(NA), "NA")
\})
\end{alltt}


The .Rd file will be created as normal, with an example section like:

\begin{alltt}\bsl{}examples\{
fib(2)

n <- 6 
fib(n)
fib("a")
fib(NA)
\}
\end{alltt}

\end{SubSection}


%
\begin{SubSection}{Usage}

Install doctest from \Rhref{https://r-universe.dev}{r-universe}:

\begin{alltt}install.packages("doctest", repos = c("https://hughjonesd.r-universe.dev", 
                                      "https://cloud.r-project.org"))
\end{alltt}


Or from CRAN:

\begin{alltt}install.packages("doctest")
\end{alltt}


Or get the development version:

\begin{alltt}devtools::install("hughjonesd/doctest")
\end{alltt}


To use doctest in your package, alter its DESCRIPTION file to add the
\code{dt\_roclet} roclet and \code{"doctest"} package to roxygen:

\begin{alltt}Roxygen: list(roclets = c("collate", "rd", "namespace", 
              "doctest::dt_roclet"), packages = "doctest") 
\end{alltt}


Then use \code{roxygen2::roxygenize()} or \code{devtools::document()} to build
your package documentation.

Doctest is \strong{[Experimental]}.
\end{SubSection}

\end{Details}
%
\begin{Author}
\strong{Maintainer}: David Hugh-Jones \email{davidhughjones@gmail.com}

\end{Author}
%
\begin{SeeAlso}
Useful links:
\begin{itemize}

\item{} \url{https://hughjonesd.github.io/doctest/}
\item{} Report bugs at \url{https://github.com/hughjonesd/doctest/issues}

\end{itemize}


\end{SeeAlso}
\inputencoding{utf8}
\HeaderA{doctest-tag}{Start a doctest}{doctest.Rdash.tag}
\aliasA{@doctest}{doctest-tag}{@doctest}
%
\begin{Description}
\AsIs{\texttt{@doctest}} starts a doctest: a code example that also contains one or more
\LinkA{testthat}{testthat} expectations.
\end{Description}
%
\begin{Details}
Use \AsIs{\texttt{@doctest}} where you would usually use \AsIs{\texttt{@examples}}. Then add
\LinkA{@expect}{@expect} and \LinkA{@expectRaw}{@expectRaw} tags beneath it to create expectations.

By default, a test labelled "Example: <object name>" is created. You
can put a different label after \AsIs{\texttt{@doctest}}:

\begin{alltt}#' @doctest Positive numbers
#'
#' x <- 1
#' @expect equal(x)
#' abs(x)
#'
#' @doctest Negative numbers
#' x <- -1
#' @expect equal(-x)
#' abs(x)
\end{alltt}


You can have more than one \AsIs{\texttt{@doctest}} tag in a roxygen block. Each doctest
will create a new test, but they will all be merged into a single Rd example.
Each doctest must contain an independent unit of code. For example, this
won't work:

\begin{alltt}#' @doctest Test x
#' @expect equal(2)
#' x <- 1 + 1
#'
#' @doctest Keep testing x
#' @expect equal(4)
#' x^2
#' # Test will error, because `x` has not been defined here
\end{alltt}


A test will only be written if the \AsIs{\texttt{@doctest}} section has at least one
\LinkA{@expect}{@expect} or \LinkA{@expectRaw}{@expectRaw} in it. This lets you change \AsIs{\texttt{@examples}} to
\AsIs{\texttt{@doctest}} in your code, without generating unexpected tests.
\end{Details}
\inputencoding{utf8}
\HeaderA{doctestExample-tag}{Add an example from a file}{doctestExample.Rdash.tag}
\aliasA{@doctestExample}{doctestExample-tag}{@doctestExample}
%
\begin{Description}
\AsIs{\texttt{@doctestExample path/to/file.R}} is a drop-in replacement for
\AsIs{\texttt{@example path/to/file.R}}. It doesn't add the contents of \code{file.R} to
the test.
\end{Description}
%
\begin{Details}
If you have complex examples you may want to store them separately.
Roxygen2 uses the \AsIs{\texttt{@example}} tag for this. \AsIs{\texttt{@doctestExample}} does the
same: it adds the contents of its file to the resulting example.
Suppose \code{man/R/example-code.R} contains the line:

\begin{alltt}2 + 2
\end{alltt}


Then the following roxygen:

\begin{alltt}#' @doctest
#'
#' @expect equal(2)
#' 1 + 1
#' @doctestExample man/R/example-code.R
\end{alltt}


will generate an example like:

\begin{alltt}1 + 1
2 + 2
\end{alltt}


At present, \AsIs{\texttt{@doctestExample}} doesn't add any code to the tests.

\AsIs{\texttt{@doctestExample}} was added in doctest 0.3.0.
\end{Details}
\inputencoding{utf8}
\HeaderA{dt\_roclet}{Create the doctest roclet}{dt.Rul.roclet}
%
\begin{Description}
You can use this in your package DESCRIPTION like this:

\begin{alltt}Roxygen: list(roclets = c("collate", "rd", "namespace", "doctest::dt_roclet"))
\end{alltt}

\end{Description}
%
\begin{Usage}
\begin{verbatim}
dt_roclet()
\end{verbatim}
\end{Usage}
%
\begin{Value}
The doctest roclet
\end{Value}
%
\begin{Examples}
\begin{ExampleCode}
## Not run: 
roxygen2::roxygenize(roclets = "doctest::dt_roclet")

## End(Not run)
\end{ExampleCode}
\end{Examples}
\inputencoding{utf8}
\HeaderA{expect-tag}{Create an expectation}{expect.Rdash.tag}
\aliasA{@expect}{expect-tag}{@expect}
\keyword{expectations}{expect-tag}
%
\begin{Description}
\AsIs{\texttt{@expect}} creates an expectation for your example code.
\end{Description}
%
\begin{Details}
Use \AsIs{\texttt{@expect}} to create a \LinkA{testthat}{testthat} expectation.

\begin{alltt}#' @doctest
#'
#' @expect equals(4)
#' 2 + 2
#'
#' f <- function () warning("Watch out")
#' @expect warning()
#' f()
\end{alltt}


The next expression will be inserted as the first
argument to the \AsIs{\texttt{expect\_*}} call.

Don't include the \code{expect\_} prefix.

If you want to include the expression in a different
place or places, use a dot \code{.}:

\begin{alltt}@expect equals(., rev(.))
c("T", "E", "N", "E", "T")
\end{alltt}


The \AsIs{\texttt{@expect}} tag and code must fit on a single line.
\end{Details}
%
\begin{SeeAlso}
Other expectations: 
\code{\LinkA{expectRaw-tag}{expectRaw.Rdash.tag}},
\code{\LinkA{snap-tag}{snap.Rdash.tag}}
\end{SeeAlso}
\inputencoding{utf8}
\HeaderA{expectRaw-tag}{Create an expectation as-is}{expectRaw.Rdash.tag}
\aliasA{@expectRaw}{expectRaw-tag}{@expectRaw}
\keyword{expectations}{expectRaw-tag}
%
\begin{Description}
\AsIs{\texttt{@expectRaw}} creates an expectation for your example code, without adding
the next expression as the subject.
\end{Description}
%
\begin{Details}
\AsIs{\texttt{@expectRaw}} creates a \LinkA{testthat}{testthat} expectation.
Unlike \LinkA{@expect}{@expect}, it doesn't insert the subsequent expression as the first
argument.

\begin{alltt}#' @doctest
#'
#' x <- 2 + 2
#' @expectRaw equals(x, 4)
#'
#' f <- function () warning("Watch out")
#' @expectRaw warning(f())
\end{alltt}


Don't include the \code{expect\_} prefix.

The \AsIs{\texttt{@expectRaw}} tag and code must fit on a single line.
\end{Details}
%
\begin{SeeAlso}
Other expectations: 
\code{\LinkA{expect-tag}{expect.Rdash.tag}},
\code{\LinkA{snap-tag}{snap.Rdash.tag}}
\end{SeeAlso}
\inputencoding{utf8}
\HeaderA{omit-tag}{Exclude example code from a test}{omit.Rdash.tag}
\aliasA{@omit}{omit-tag}{@omit}
\aliasA{@resume}{omit-tag}{@resume}
\aliasA{resume-tag}{omit-tag}{resume.Rdash.tag}
%
\begin{Description}
\AsIs{\texttt{@omit}} excludes example code from a test until the next tag.
Use \AsIs{\texttt{@resume}} to restart including code without creating an expectation.
\end{Description}
%
\begin{Details}
Use \AsIs{\texttt{@omit}} to avoid redundant or noisy code:

\begin{alltt}#' @doctest
#'
#' @expect equal(0)
#' sin(0)
#'
#' @omit
#' curve(sin(x), 0, 2 * pi)
#'
#' @expect equal(1)
#' cos(0)
\end{alltt}


\AsIs{\texttt{@omit}} is separate from \AsIs{\texttt{\bsl{}donttest}} and \AsIs{\texttt{\bsl{}dontrun}} tags in Rd files. This
allows you to test code that would cause an error if run by R CMD CHECK. If
you also want R CMD CHECK to skip your code, you should use \AsIs{\texttt{\bsl{}donttest\{\}}}
separately (see
\Rhref{https://cran.r-project.org/doc/manuals/r-release/R-exts.html\#Writing-R-documentation-files}{writing R extensions}).

Remember that the main purpose of examples is to document your package for
your users. If your code is getting too different from your example, consider
splitting it off into a proper test file. You can do this by renaming it and
deleting the \AsIs{\texttt{Generated by doctest}} comment.
\end{Details}
\inputencoding{utf8}
\HeaderA{snap-tag}{Create a snapshot test}{snap.Rdash.tag}
\aliasA{@snap}{snap-tag}{@snap}
\keyword{expectations}{snap-tag}
%
\begin{Description}
\AsIs{\texttt{@snap}} creates a
\Rhref{https://testthat.r-lib.org/articles/snapshotting.html}{snapshot test}
for your example. It is shorthand for \AsIs{\texttt{@expect snapshot()}}.
\end{Description}
%
\begin{Details}
Often, examples show complex output to the user. If this output changes,
you want to check that it still "looks right". Snapshot tests help by
failing when the output changes, and allowing you to review and approve
the new output.

\begin{alltt}#' @doctest
#'
#' @snap
#' summary(lm(Petal.Width ~ Species, data = iris))
\end{alltt}

\end{Details}
%
\begin{SeeAlso}
Other expectations: 
\code{\LinkA{expect-tag}{expect.Rdash.tag}},
\code{\LinkA{expectRaw-tag}{expectRaw.Rdash.tag}}
\end{SeeAlso}
\inputencoding{utf8}
\HeaderA{testRaw-tag}{Add a line of code to the test}{testRaw.Rdash.tag}
\aliasA{@testRaw}{testRaw-tag}{@testRaw}
%
\begin{Description}
\AsIs{\texttt{@testRaw}} adds an arbitrary line of code to your test, without including it
in the .Rd example.
\end{Description}
%
\begin{Details}
\AsIs{\texttt{@testRaw}} adds an arbitrary line of code to your test:

\begin{alltt}#' @doctest
#' @testRaw skip_on_cran("Takes too long")
#' @expect equal(6765)
#' fibonacci(20)
\end{alltt}


Unless your doctest has at least one \LinkA{@expect}{@expect} or \LinkA{@expectRaw}{@expectRaw} tag, no test
will be created. So use those tags, not \AsIs{\texttt{@testRaw}}, to add expectations.

Remember that the main purpose of examples is to document your package for
your users. If your code is getting too different from your example, consider
splitting it off into a proper test file. To do this, rename the
file in \code{tests/testthat}, and deleting the \AsIs{\texttt{Generated by doctest}} comment.
\end{Details}
\inputencoding{utf8}
\HeaderA{test\_doctests}{Test doctests in a package}{test.Rul.doctests}
%
\begin{Description}
This is a utility function to run doctests in a local source package.
It calls \code{\LinkA{testthat::test\_local()}{testthat::test.Rul.local()}}.
\end{Description}
%
\begin{Usage}
\begin{verbatim}
test_doctests(path = ".", ...)
\end{verbatim}
\end{Usage}
%
\begin{Arguments}
\begin{ldescription}
\item[\code{path}] Path to package

\item[\code{...}] Passed to \code{\LinkA{testthat::test\_local()}{testthat::test.Rul.local()}}.
\end{ldescription}
\end{Arguments}
%
\begin{Value}
The result of \code{\LinkA{testthat::test\_local()}{testthat::test.Rul.local()}}.
\end{Value}
%
\begin{Examples}
\begin{ExampleCode}
## Not run: 
  test_doctests()

## End(Not run)
\end{ExampleCode}
\end{Examples}
\printindex{}
\end{document}
